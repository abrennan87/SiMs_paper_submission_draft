\documentclass[a4paper,11pt]{article}
\pdfoutput=1 % if your are submitting a pdflatex (i.e. if you have
             % images in pdf, png or jpg format)

\usepackage{jheppub} % for details on the use of the package, please
                     % see the JHEP-author-manual

% additional packages
\usepackage{lineno}
\usepackage{caption}
\usepackage{subcaption}		% needed for making subfigures
\usepackage{multirow}		% for multirow capabilities in tables
\usepackage{tikz}		% for drawing Feynman diagrams directly
\usepackage{array}      	% needed for defining new column type
\usepackage{xcolor}
\usepackage{amsmath}
\usepackage{rotating}   % for sidewaysfigure
\usepackage{afterpage}  % to make sidewaysfigures appear in the paper rather than at the end
%\usepackage{parskip}		% used to remove paragraph indentation

\usepackage{hyperref}

% tikz definitions
\tikzstyle arrowformat=[scale=1.5]
\tikzstyle scalar =[dashed]
\tikzstyle mom=[draw=blue]
\tikzset{
gluon/.style={decorate, draw=black, decoration={coil, amplitude = 3.5pt, segment length=4.8pt}},
photon/.style={decorate, draw=black, decoration={snake=coil, amplitude = 2pt, segment length=9pt}},
zbo/.style={decorate, draw=black, decoration={snake=coil}},
fermion/.style={draw=black, postaction={decorate}, decoration={markings, mark = at position 0.55 with {\arrow[arrowformat]{>}}}},
antifermion/.style={draw=black, postaction={decorate}, decoration={markings, mark = at position 0.55 with {\arrow[arrowformat]{<}}}}
}
\usetikzlibrary{patterns}
\usetikzlibrary{arrows}
\usetikzlibrary{decorations.pathmorphing,decorations.markings, decorations.pathreplacing, trees}

% shortcuts
\newcommand{\MG}{M{\footnotesize AD}G{\footnotesize RAPH}5 }
\newcommand{\MGnospace}{M{\footnotesize AD}G{\footnotesize RAPH}5}
\newcommand{\PYTHIA}{P{\footnotesize YTHIA} 8 }
\newcommand{\PYTHIAnospace}{P{\footnotesize YTHIA} 8}
\newcommand{\FASTJET}{F{\footnotesize AST}J{\footnotesize ET}}
\newcommand\T{\rule{0pt}{2.6ex}}       				% Top strut - useful for additional space in tables
\newcommand\B{\rule[-1.2ex]{0pt}{0pt}} 				% Bottom strut - see above
\newcommand{\met}{\ensuremath{E_{\mathrm{T}}^{\mathrm{miss}}}}		% shorthand for missing ET
\newcommand{\metvec}{\vec{E}_{\mathrm{T}}^{\mathrm{miss}}}	% shorthand for missing ET vector
\newcommand{\mDM}{m_{\mathrm{DM}}}
\newcommand{\mX}{m_{\chi}}
\newcommand{\Mmed}{M_{\mathrm{med}}}
\newcommand{\Mstar}{M_{\star}}					% shorthand for M*
\newcommand{\monoZ}{mono-$Z$(lep) }
\newcommand{\monojet}{mono-jet }
\newcommand{\monoWZ}{mono-$W/Z$(had) }
\newcommand{\monoX}{mono-$X$ }
\newcommand{\monoXnospace}{mono-$X$}
\newcommand{\gq}{g_q}
\newcommand{\gX}{g_{\chi}}
\newcommand{\sqrtgqgX}{\sqrt{g_q g_{\chi}}}
\newcommand{\gqX}{g_{q \chi}}

\newcommand{\sigv}{\langle\sigma v\rangle} %annihilation cross section

\newcolumntype{C}[1]{>{\centering\let\newline\\\arraybackslash\hspace{0pt}}m{#1}}       % used for centering columns of the same width

%\linenumbers		% turn on line numbers

%%%%%%% title, paper structure %%%%%%%%

\title{Collide and Conquer: Constraints on Simplified Dark Matter Models using Mono-$X$ Collider Searches}

\author[a,1]{A. J. Brennan,\note{Corresponding author.}}
\author[a]{M. F. McDonald,}
\author[b]{J. Gramling}
\author[c]{and T. D. Jacques}

%% The "\note" macro will give a warning: "Ignoring empty anchor..."
%% you can safely ignore it.

\affiliation[a]{ARC Centre of Excellence for Particle Physics at the Terascale \\ School of Physics, The University of Melbourne, Victoria 3010, Australia}
\affiliation[b]{Universit\'{e} de Gen\`{e}ve, Quai E. Ansermet 24, 1211 Gen\`{e}ve 4, Switzerland}
\affiliation[c]{SISSA/ISAS, via Bonomea 265, 34136 Trieste, Italy}

% e-mail addresses: one for each author, in the same order as the authors
\emailAdd{amelia.brennan@coepp.org.au}
\emailAdd{millie.mcdonald@coepp.org.au}
\emailAdd{johanna.gramling@cern.ch}
\emailAdd{thomas.jacques@sissa.it}

\abstract{The use of simplified models as a tool for interpreting dark matter collider searches has become increasingly prevalent, and while early Run II results are beginning to appear, we look to see what further information can be extracted from the Run I dataset. We consider three `standard' simplified models that couple quarks to fermionic singlet dark matter: an $s$-channel vector mediator with vector or axial-vector couplings, and a $t$-channel scalar mediator. Upper limits on the couplings are calculated and compared across three alternate channels, namely mono-jet, mono-$Z$ (leptonic) and mono-$W/Z$ (hadronic). The strongest limits are observed in the mono-jet channel, however the computational simplicity and absence of significant $t$-channel model width effects in the mono-boson channels make these a straightforward and competitive alternative. We also include a comparison with relic density and direct detection constraints.}

\begin{document}

\hfill
SISSA 18/2016/FISI

\maketitle
\flushbottom

\section{Introduction}
\label{sec:sec1}
Simplified models have emerged as a powerful tool for the interpretation of collider, direct and indirect detection signals of dark matter (DM). Previously, ATLAS and CMS searches for DM were conducted within the context of both Effective Field Theories (EFTs) \cite{Aad:1363019, ATLAS-CONF-2012-147, CMS-PAS-EXO-12-048, Abdallah:1472683} and full UV-complete theories such as Supersymmetry \cite{Aad:2012ms, Aad:2012fqa, Aad:2014wea, SUSY_official_paper}. The latter approach, though well-motivated, is typified by a broad parameter space and generally yields results which are insensitive to the wider class of DM models. EFT constraints, in comparison, are applicable to a broad range of models and rely on the specification of only a small set of parameters, namely the suppression scale, $\Mstar$, and the DM mass, $\mDM$ \cite{DMCons2}.

In the EFT framework, interactions between the dark and Standard Model (SM) sector are parametrised by a set of higher-dimensional effective operators that arise when the mass of the mediating particle is assumed to be significantly larger than the momentum transferred in a given interaction. Where this condition is not fulfilled, the EFT prescription can produce constraints which detour dramatically from those of the associated UV-complete model \cite{Bai:2010hh, DMCons2, Fox:2011fx, Graesser:2011vj, An:2011ck}. This is less important in direct detection experiments where the momentum transferred in the scattering of DM particles with heavy nuclei is generally of the order of tens of MeV \cite{EFTDM, DMCons3}, or in indirect searches where the annihilations of non-relativistic DM particles in the galactic halo occur with momentum transfers of order $\mDM$. However, for hadron collider searches, where the accessible center of mass energy of two colliding baryons may be sufficient to produce the mediator on-shell, the range of validity of the EFT prescription is significantly diminished. Indeed, recent works have quantitatively shown that the EFT approach can break down as a valid interpretation of data collected during the $\sqrt{\hat{s}} =$ 8 TeV Run I of the Large Hadron Collider (LHC) \cite{Buchmueller:2013dya,ValidEFT, ValidEFT_part2, ValidEFT_part3}. In light of this, simplified models have become the preferred tool for the interpretation of collider DM searches \cite{DM_MET_LHC, DMOxfordReport, DMForumReport, Harris:2014hga,Buchmueller:2014yoa}.

In a nutshell, a simplified model (SiM) arises when the heavy mediator which was integrated out in the EFT framework is reintroduced. Like EFTs, SiMs admit the comparison of results obtained in the different avenues of dark matter study \cite{DiFranzo:2013vra, Buckley:2014fba} and are defined by a relatively small set of parameters - namely $\mDM$, the mass of the mediator $\Mmed$, and the SM-mediator and DM-mediator coupling strengths, $\gq$ and $\gX$ (or $\gqX$ in the case of a single, SM-DM-mediator coupling). Unlike EFTs, constraints calculated within the context of a SiM are valid across a broad energy range.

In this paper, we examine a phenomenologically distinct set of SiMs. In particular, we place constraints on the SiMs corresponding to the simplest UV-completions of the D5 (vector) and D8 (axial-vector) effective operators in the $s$-channel\footnote{The D5 and D8 operators form a nice starting point in the analysis of SiMs as they have been studied exhaustively in the past (see refs.~\cite{Aad:1363019, ATLAS-CONF-2012-147, CMS-PAS-EXO-12-048, Buckley:2013jwa, Abdallah:1472683, MonoX, ValidEFT, ValidEFT_part2, ValidEFT_part3} among others). This attention is motivated by the fact that collider limits for the D5 (D8) operator can be readily transformed into limits on spin-independent (spin-dependent) DM-nucleon scattering and vice versa.}. We also include a case in which a scalar mediator is exchanged in the $t$-channel, motivated by its analog of squark exchange in Supersymmetry. In the heavy mediator limit, the operator describing this model can be rearranged via a Fierz transformation into a combination of operators D5 to D8.

The models are constrained using public results from \monoX + missing transverse energy ($\met$) searches conducted by the ATLAS Collaboration. Specifically, we focus on searches where $X$ is either a parton (manifesting in the detector as a narrow-radius jet), a leptonically-decaying $Z$ boson (manifesting as two opposite-sign same-flavor leptons), or a hadronically-decaying $W$ or $Z$ boson (manifesting as a large-radius jet). The purpose of this work is to strengthen existing SiM limits using the full 20.3 $fb^{-1}$ of Run I ATLAS data, and to explore an enhanced phase space with respect to the mediator and DM masses and the relative strength of the couplings to the visible and dark sectors. We choose to treat the mediator width as the minimal value naturally arising, which is more realistic than a fixed width. Lastly, we provide a cross-check and comparison of the performance of the three targeted collider detection channels, and compare against relic density and direct detection constraints.

The remainder of the paper is organised as follows. Section \ref{sec:sec2} contains a compendium of the SiMs chosen for analysis and the associated collider phenomenology. Section \ref{sec:sec3} outlines the techniques used to recast \monoX + $\met$ limits on the visible cross-section for any new physics process into constraints on SiMs, and specifically on the couplings $\gq$ and $\gX$. Lastly, our results are presented in section \ref{sec:sec4} along with a discussion of the implications of this work. Appendices \ref{Appendix_limitsetting} and \ref{Appendix_validation} include  details of the limit setting and analysis validation procedures.


\section{Simplified model phenomenology}
\label{sec:sec2}
\subsection{Model descriptions}
We begin with a short set of assumptions: that the DM particle, $\chi$, is a weakly interacting Dirac fermion, that it is a singlet under the SM, and that it is the lightest stable new particle.
Additionally, the new sector is assumed to couple only to the SM quarks. While possible couplings to SM leptons or gluons have been studied elsewhere (see, for example, ref.~\cite{Fox:2011fx, SiM_gluons}), they are beyond the scope of this paper. The nature of the mediating particle then results from these assumptions: in the $s$-channel it is chosen to be a vector particle which then must also be a SM singlet, denoted $\xi$, while in the $t$-channel we chose a scalar particle which is necessarily charged and coloured, and labelled $\phi$.

The $s$-channel models chosen for analysis are $Z'$-type models characterised by vector ($sV$) or axial-vector ($sA$) couplings to both the dark and SM sectors. They are described by the following interaction Lagrangians:
\begin{equation}
\label{L_int_sV}
\mathcal{L}_{sV} \supset - \xi_{\mu}\left[ \sum\limits_{q} \gq\bar{q}\gamma^{\mu}q + \gX\bar{\chi}\gamma^{\mu}\chi\right],
\end{equation}
\begin{equation}
\label{L_int_sA}
\mathcal{L}_{sA} \supset - \xi_{\mu}\left[\sum\limits_{q} \gq\bar{q}\gamma^{\mu}\gamma_{5}q + \gX\bar{\chi}\gamma^{\mu}\gamma_{5}\chi\right],
\end{equation}
where the sum is over all quarks. This is a simple extension of the standard model and has been studied extensively \cite{Buchmueller:2014yoa, Heisig:2015ira,Blennow:2015gta,Lebedev:2014bba, Alves:2015pea, Alves:2013tqa, Alves:2015mua, An:2012va, An:2012ue, Frandsen:2012rk, Arcadi:2013qia, Shoemaker:2011vi, Frandsen:2011cg, Gondolo:2011eq, Fairbairn:2014aqa, Harris:2014hga, NordstromSVD, Bell:2015rdw, Chala:2015ama, Kahlhoefer:2015bea}.
For the couplings $\gq$ and $\gX$ to remain within the perturbative regime, they are required to satisfy $\gq,\gX \leq 4\pi$, though stronger perturbativity requirements do exist \cite{ValidEFT}.

The $t$-channel model (abbreviated $tS$) is primarily motivated by analogy with a common aspect of Supersymmetric models: neutralino DM interacting with the SM sector via $t$-channel exchange of a squark \cite{SUSYDM}, and has been studied within the context of collider searches by a number of groups \cite{DiFranzo:2013vra, Bai:2013iqa, An:2013xka, Chang:2013oia, Zurek:tchannel, Garny:2015wea,  Garny:2014waa, Bell:2011if, Bell:2015rdw}. Note that in this Supersymmetric scenario the DM particle is a Majorana fermion. The collider phenomenology of a SiM in which $\chi$ is of Majorana type is kinematically identical to the corresponding Dirac case (requiring multiplication of the cross-section by a simple factor in order to compute limits) and so Majorana DM is not covered here\footnote{The exception being in the validation of the \monoZ channel, see Sec. \ref{monoZ_validation}.}.

In the $tS$ model, the mediator is allowed to couple to either the left or right-handed quarks as an SU(2) doublet or singlet respectively. Since the LHC is insensitive to the chirality of the quarks, we assume for simplicity that $\phi$ couples to the left-handed quarks only, and is itself an SU(2) doublet, allowing radiation of a $W$ boson. To avoid different couplings to quarks of different generations, and to remain in step with the DM Forum recommendations \cite{DMForumReport}, we include three generations of mediator doublets $\phi_i$, with equal masses and couplings. The interaction Lagrangian for this model is then:
\begin{equation}
\label{L_int_tS}
\mathcal{L}_{tS} \supset \sum_{i} \gqX \bar{Q_i} P_R \phi_i \chi + {\rm h.c.},
\end{equation}
where the sum is over the three quark doublets, $\gqX$ is the DM-quark coupling (equal for each generation), and $P_R$ is the usual chiral projection operator.

\subsection{The mono-$X$ + $\met$ signatures}
The \monoX + $\met$ signal (abbreviated to \monoXnospace) is a popular collider signal in the search for new physics, particularly in the search for dark matter. Since DM particles are not expected to interact with detector material, they appear as missing transverse energy when balanced against a visible object, $X$, that is radiated from the initial or intermediate state. For the $s$-channel SiMs discussed above, only initial-state radiation is permitted; see figs.~\ref{fig:FD_sV_gluonISR} and \ref{fig:FD_sV_WZISR} for examples. For the $tS$ model, radiation of a gluon or electroweak (EW) boson is permitted both from initial state partons (fig.~\ref{fig:FD_tS_gluonISR}) or from the mediator (fig.~\ref{fig:FD_tS_WZmediator}).

The most likely scenario at the LHC is the production of a jet alongside the invisible $\chi$ pair, as a result of the strong coupling and prevalence of partons in the initial state. However, to fully exploit the potential of the ATLAS detector to record and identify a vast array of particle types, we also consider two additional channels. Firstly, we take advantage of the relative cleanliness and simplicity of leptons in the leptonically-decaying mono-$Z$ ($\rightarrow \ell^+ \ell^-)$ channel. We also take advantage of the large hadronic branching fraction, and developing jet-identification techniques for boosted EW bosons, in the hadronically-decaying mono-$W/Z$ ($\rightarrow jj)$ channel\footnote{In addition, one of the first Run II dark matter search results from ATLAS was from this channel \cite{monoWZ_run2}, released during the preparation of this paper.}. In both cases, the large multi-jet background is reduced, and complications in jet production such as parton-matching can be ignored, making these an interesting alternative to the \monojet channel where speed, efficiency and a reduction in jet-associated uncertainties may make up for a loss in sensitivity.

\begin{figure}[t]
  \centering
  \begin{subfigure}[b]{0.45\textwidth}
    \centering
    \resizebox{\linewidth}{!}{
      \begin{tikzpicture}
        \draw[fermion] (-1.5,1.5)node[left]{$q$} --(-0.75,0.75);
        \draw[gluon] (-0.75,0.75) -- (0,1.5)node[right]{$g$};
        \draw[fermion] (-0.75,0.75) -- (0,0);
        \draw[antifermion] (-1.5,-1.5)node[left]{$\bar{q}$} --(0,0);
        \draw[fill] (0,0) circle [radius=0.0]node[left]{$\gq\mbox{ }$};
        \draw[photon] (0,0) --node[above]{$\xi$} (2,0);
        \draw[fermion] (2,0) -- (3.5,1.5)node[right]{$\chi$};
        \draw[antifermion] (2,0) --(3.5,-1.5)node[right]{$\bar{\chi}$};
        \draw[fill] (2,0) circle [radius=0.0]node[right]{$\mbox{ }\gX$};
      \end{tikzpicture}
    }
    \caption{}
    \label{fig:FD_sV_gluonISR}
  \end{subfigure}
  \begin{subfigure}[b]{0.45\textwidth}
    \centering
    \resizebox{\linewidth}{!}{
      \begin{tikzpicture}
        \draw[fermion] (-1.5,1.5)node[left]{$q$} --(-0.75,0.75);
        \draw[photon] (-0.75,0.75) -- (0,1.5)node[right]{$W/Z$}; %\draw[dashed] (-0.75,0.75) -- (0,1.5)node[right]{X};
        \draw[fermion] (-0.75,0.75) -- (0,0);
        \draw[antifermion] (-1.5,-1.5)node[left]{$\bar{q}$} --(0,0);
        \draw[fill] (0,0) circle [radius=0.0]node[left]{$\gq\mbox{ }$};
        \draw[photon] (0,0) --node[above]{$\xi$} (2,0);
        \draw[fermion] (2,0) -- (3.5,1.5)node[right]{$\chi$};
        \draw[antifermion] (2,0) --(3.5,-1.5)node[right]{$\bar{\chi}$};
        \draw[fill] (2,0) circle [radius=0.0]node[right]{$\mbox{ }\gX$};
      \end{tikzpicture}
    }
    \caption{}
    \label{fig:FD_sV_WZISR}
  \end{subfigure}
  \begin{subfigure}[b]{0.4\textwidth}
    \centering
    \resizebox{\linewidth}{!}{
      \begin{tikzpicture}
        \draw[fermion] (-2,1.)node[left]{$q$} --(-1,1.);
        \draw[fermion] (-1,1.) --(0,1.);
        \draw[gluon] (-1,1.) --(0,2.)node[right]{$g$};
        \draw[antifermion] (-2,-1)node[left]{$\bar{q}$} --(0,-1);
        \draw[fill] (0,1.) circle [radius=0.0]node[above]{$\gqX$};
        \draw[dashed] (0,1.) -- node[left]{$\phi_{q}$}(0,-1);
        \draw[fermion] (0,1.) -- (2,1.)node[right]{$\chi$};
        \draw[antifermion] (0,-1) --(2,-1)node[right]{$\bar{\chi}$};
        \draw[fill] (0,-1) circle [radius=0.0]node[below]{$\gqX$};
      \end{tikzpicture}
    }
    \caption{}
    \label{fig:FD_tS_gluonISR}
  \end{subfigure}
  \hspace{1cm}
  \begin{subfigure}[b]{0.4\textwidth}
    \centering
    \resizebox{\linewidth}{!}{
      \begin{tikzpicture}
        \draw[fermion] (-2,1.)node[left]{$q$} --(0,1.);
        \draw[antifermion] (-2,-1)node[left]{$\bar{q}'$} --(0,-1);
        \draw[fill] (0,1.) circle [radius=0.0]node[above]{$\gqX$};
        \draw[dashed] (0,1.) -- node[left]{$\phi_{q}$}(0,0.25);
        \draw[photon] (0,0.) -- (1.5, 0.)node[right]{$W/Z$};
        \draw[dashed] (0,0.) -- node[left]{$\phi_{q'}$}(0,-1);
        \draw[fermion] (0,1.) -- (2,1.)node[right]{$\chi$};
        \draw[antifermion] (0,-1) --(2,-1)node[right]{$\bar{\chi}$};
        \draw[fill] (0,-1) circle [radius=0.0]node[below]{$\gqX$};
      \end{tikzpicture}
    }
    \caption{}
    \label{fig:FD_tS_WZmediator}
  \end{subfigure}
  \caption{Representative dark matter pair-production processes with a gluon or $W/Z$ boson in the final state for the $s$-channel (a,b) and $t$-channel (c,d) models. Note that other diagrams are possible, including initial state radiation of a gauge boson, and internal bremsstrahlung of a gluon.}
  \label{allchannel_sig_phen}
\end{figure}

\subsection{Mass and coupling points}
A representative set of dark matter and mediator masses, listed in table \ref{Mass_coup_points}, are chosen for study in each detection channel. DM masses of 3, 30 and 300 GeV are also included in the \monoZ channel, where ease of production permits higher granularity in the mass phase space. All $(\mX, \Mmed)$ combinations are allowed in the $sV$ and $sA$ models, while in the $tS$ model $\Mmed$ must be greater than $\mX$ to ensure stability of the DM particle. The couplings $\gq$ and $\gqX$ are set to unity, while the DM-mediator coupling in the $s$-channel models, $\gX$, is varied from 0.2 to 5. The mediator masses are chosen to cover a broad range of parameter space and to coincide with predominantly three regimes: (near-)degenerate ($\Mmed \approx \mX$), on-shell ($\Mmed \geq 2 \mX$) and off-shell ($\Mmed < 2 \mX$). Note that these diagrams do not comprise a comprehensive set.

\begin{table}
\centering
\begin{tabular}{C{3cm} | C{3cm} | C{1.5cm}  C{1.5cm} | C{3cm}}
\hline
\hline
\multirow{2}{*}{$\mX$ [GeV]} & \multirow{2}{*}{$\Mmed$ [GeV]} & \multicolumn{2}{c|} {$s$-channel} & $t$-channel \T \B \\
& & $\gq$ & $\gX$ & $\gqX$ \T \B\\
\hline
1, (3), 10, (30), 100, (300), 1000 & 1, 2, 10, 20,  100, 200, 1000, 2000 & 1 & 0.2, 0.5, 1, 2, 5 & 1 \T \B  \\
\hline
\hline
\end{tabular}
\caption{Mass and coupling points chosen for the analysis of simplified dark matter models. Values in brackets are only included in the \monoZ channel. The mediator masses are primarily representative of three regimes: (near-)degenerate ($\Mmed \approx \mX$), on-shell ($\Mmed \geq 2 \mX$) and off-shell ($\Mmed < 2 \mX$). For the $t$-channel model, $\Mmed > \mX$ is required to ensure stability of the DM particle.}
\label{Mass_coup_points}
\end{table}

\subsection{Treatment of the width}
\label{width_effects}
An important factor when considering SiMs is to ensure that the mediator width is treated appropriately, as it impacts both the cross-section calculation and, in some cases, the kinematic behaviour of the model.

Following the DM Forum recommendations \cite{DMForumReport}, we use the minimal width, allowing coupling to all kinematically accessible quarks. We assume minimal flavour violation, which implies a universal coupling to all quark flavours. The minimum width for each model is given by\footnote{It is possible that the mediator may decay to other SM or BSM particles \cite{Harris:2014hga}, but this is not expected to have a large effect on the kinematic distribution as long as the width remains relatively small \cite{DMForumReport}.}:

\begin{eqnarray}
    \Gamma_{sV} \, &=& \,  \frac{\gX^2 M}{12\pi}\left(1 + \frac{2 \mX^{2}}{M^{2}}\right)\left(1 - \frac{4 \mX^{2}}{M^{2}}\right)^{\frac{1}{2}} \Theta(M-2 \mX) \nonumber\\
                  && + \sum_{\substack{q}}\frac{\gq^2M}{4\pi}\left(1 + \frac{2m_{q}^{2}}{M^{2}}\right)\left(1 - \frac{4m_{q}^{2}}{M^{2}}\right)^{\frac{1}{2}} \Theta(M-2m_q)\\[5pt]
    \Gamma_{sA} \, &=& \,  \frac{\gX^2 M}{12\pi}\left(1 - \frac{4 \mX^{2}}{M^{2}}\right)^{\frac{3}{2}} \Theta(M-2 \mX) \nonumber\\
                  && + \sum_{\substack{q}}\frac{\gq^2 M}{4\pi}\left(1 - \frac{4m_{q}^{2}}{M^{2}}\right)^{\frac{3}{2}} \Theta(M-2m_q) \\[5pt]
    \Gamma_{tS} \, &=& \,  \sum_{\substack{q}} \frac{\gqX^2M}{16\pi}\left(1 - \frac{m_{q}^{2}}{M^{2}} - \frac{\mX^{2}}{M^{2}}\right) \nonumber\\
                  && \times \sqrt{\left(1 - \frac{m_{q}^{2}}{M^{2}} + \frac{\mX^{2}}{M^{2}}\right)^{2} - 4\frac{\mX^{2}}{M^{2}}} \,\, \Theta(M-m_q- \mX)
\end{eqnarray}

We can take advantage of the fact that for each point in ($\mX$, $\Mmed$) phase space, the mediator width (and therefore the couplings) do not greatly affect a model's kinematic behaviour (with the notable exception of the $tS$ model in the \monojet channel). This is demonstrated in fig.~\ref{fig:MET_dists}, where we plot a simplified $\met$ distribution (as a proxy for the full selection in each analysis) for the $sV$ (representing both the $sV$ and $sA$ models) and $tS$ models for two mass points and a demonstrative set of couplings such that $\Gamma < \Mmed/2$. The $\met$ distribution is predominantly independent of the mediator width for the $s$-channel models in the \monojet channel, and all models in the \monoZ\footnote{In this discussion, the \monoWZ channel can be assumed to follow the same logic as for the \monoZ channel.} channel. However, there is a clear variation in the kinematic behaviour of the tS model in the \monojet channel, which can be attributed to additional diagrams (accessible only in this channel) featuring a gluon in the initial state and subsequently allowing the mediator to go on-shell. In this scenario, when the resulting quark and DM particle are both small compared to the mediator mass, they share equally its energy leading to a peak in the $\met$ distribution at approximately half the mediator mass.

In the cases where the kinematic distribution is independent of the width, we assume that the impact of the selection cuts in each channel is unchanged by the couplings. In this case, the following relations approximately hold:

\begin{equation}
  \sigma \propto
  \begin{cases}
      \gq^2 \gX^2 / \Gamma & \mathrm{ if } \, \Mmed \geq 2 \mDM\\
      \gq^2 \gX^2 & \mathrm{ if } \, \Mmed < 2 \mDM
  \end{cases}
  \label{eq:sigma_propto_couplings_schan}
\end{equation}
in the $sV$ and $sA$ models  \cite{NordstromSVD}, and

\begin{equation}
  \sigma \propto \gqX^4
  \label{eq:sigma_propto_couplings_tchan}
\end{equation}
in the $tS$ model. When valid, these approximations allow us to greatly simplify our limit calculations, and for this reason, we restrict our primary results to regions of parameter space where $\Gamma/\Mmed < 0.5$ (see app.~\ref{Appendix_limitsetting} for further details of the limit-setting calculation).

The generator treatment of the mediator as a Breit-Wigner propagator, rather than a true kinetic propagator, breaks down for large widths \cite{An:2012va,NordstromSVD}. More problematically, it was noted by refs.~\cite{NordstromSVD,An:2012va} that the Breit-Wigner propagator breaks down in the $\mDM \gg \Mmed$ region even if $\Gamma/\Mmed$ is small. To correct for this we follow ref.~\cite{NordstromSVD}, and rescale the cross-section in the $\mDM > \Mmed$ region by a factor which takes into account the error introduced by the use of a Breit-Wigner propagator by the generator. The factor is found by convolving the PDF with both the kinetic and Breit-Wigner propagators in turn and taking the ratio at each mass point. We approximate the kinetic propagator by making the substitution $\Mmed \Gamma(\Mmed) \rightarrow s \Gamma(\sqrt{s}) / \Mmed$ in the Breit-Wigner propagator.

A full study of the $tS$ model within the \monojet channel, where altering the coupling can lead to changed kinematic behaviour, has been performed elsewhere \cite{Zurek:tchannel}, and requires the production of individual samples for each coupling point. This, combined with the challenges associated with including differing orders of $\alpha_s$, make the generation process computationally expensive compared to the \monoZ and \monoWZ channels. We therefore exclude an analysis of the $tS$ model in the \monojet channel in this work.

\begin{figure}[t]
  \begin{center}
    \includegraphics[width=0.495\textwidth]{figures/MET_monojet_SVD.pdf}
    \includegraphics[width=0.495\textwidth]{figures/MET_monojet_TSD.pdf}
    \includegraphics[width=0.495\textwidth]{figures/MET_monoZ_SVD.pdf}
    \includegraphics[width=0.495\textwidth]{figures/MET_monoZ_TSD.pdf}
    \caption{The $\met$ distribution of the $sV$ and $tS$ models in the \monojet and \monoZ channels, for some exemplary masses. The parameter $\mu$ is defined as $\Gamma / \Mmed$, and is used to demonstrate the impact of a changing width; the $tS$ model in the \monojet channel shows a clear width-dependence. The widths are obtained with couplings of 0.1, 1 and 5 where $\mu < 0.5$ remains true.}
    \label{fig:MET_dists}
  \end{center}
\end{figure}


\section{Recasting mono-$X$ constraints}
\label{sec:sec3}
\input{Reanalysis.tex}

\section{Results and discussion}
\label{sec:sec4}
\input{Results.tex}

\section{Conclusion}
\label{sec:sec5}
In this paper we have examined a set of three simplified dark matter models, extracting constraints from ATLAS Run I mono-$X$ plus missing energy searches featuring the associated production of a mono-jet, mono-$Z(\rightarrow$ leptons), or mono-$W/Z (\rightarrow$ hadrons). We explored a phase space where both the DM and mediator masses span $\mathcal{O}$(GeV) to $\mathcal{O}$(TeV), and considered ratios of $\gX / \gq$ of 0.2, 0.5, 1, 2 and 5 in the $s$-channel models.

Rather than setting limits in the $\Mmed - \mDM$ plane for a fixed value of the coupling strength, we instead constrained the coupling strength as a function of both $\Mmed$ and $\mDM$ in a 3D plane. Whilst this approach necessitates the introduction of some approximations, it also allows for a thorough examination of the interplay between the DM production cross-section and the free parameters of the models.

As expected, the \monojet channel is found to yield the strongest limits on vector and axial-vector SM and DM couplings to a vector mediator exchanged in the $s$-channel. This channel is also found to perform well for small values of $\gX$. The limits obtained in the \monoZ channel, in comparison, are generally weaker by a factor of a few, while the \monoWZ results are weaker again. This is partly due to our conservative estimations of the systematic uncertainties and partly due to limited statistics resulting from a harder $\met$ selection cut. The width effects associated with the $t$-channel exchange of an SU(2) doublet scalar mediator are observed to vanish in both the \monoZ and \monoWZ channels, greatly simplifying the analysis of this model and confirming these as straightforward and competitive channels for future collider DM detection.

Where the axial-vector model is not excluded by perturbative unitarity requirements, we find the coupling limits to be on par with those of the vector model within each analysis channel. Weaker limits are found for the $t$-channel model, a result of cross-section suppression not present in the $s$-channel models.

Finally, we compared our limits to constraints from relic density and direct detection; although each search is subject to a different set of assumptions, this demonstrates the complementarity and impressive reach of simplified models as a tool for the interpretation of collider DM searches. We eagerly await the improved constraints expected from Run II of the LHC.


\section{Acknowledgements}
\label{sec:sec6}
A.J.B. and M.F.M. were supported by the Australian Research Council. J.G.~was supported by SNF (grant 200020\_156083). We thank Karl Nordstr{\"o}m for discussions on the cross-section reweighting, Brian Petersen, Steven Schramm, Rebecca Leane, Nicole Bell and Elisabetta Barberio for further helpful discussions, and Sean Crosby for technical support.


\appendix

\section{Limit setting strategy}
\label{Appendix_limitsetting}
\input{AppendixA.tex}

\section{Validation of signal simulation and event selection procedures}
\label{Appendix_validation}
\subsection{Mono-jet channel}
\label{monojet_validation}
The signal generation and selection procedures for the \monojet channel are validated via reproduction of the ATLAS limits on $\Mstar \equiv \Mmed / \sqrtgqgX$, for the $s$-channel vector SiM. A comparison of SR7\footnote{We use this signal region as it is the only one for which ATLAS limits are provided.} limits for a representative sample of mediator masses with $\mX = $ 50 GeV, $\Gamma = M/8\pi$ and $\sqrtgqgX = 1$ is presented in table \ref{M_star_limits_monojet}. In general, good agreement is observed between the ATLAS and reproduced limits, with a maximum difference of 12\%. We note that a discrepancy of a few percent is expected given the differences in signal simulation. For example, the simplified matching procedure discussed in detail in Sec~\ref{matching_procedure} introduces an additional uncertainty of approximately 25\% for events with $\met > 350$ GeV when compared to the approach utilised by the ATLAS \monojet group. Further uncertainties are introduced by the jet smearing approximation used in place of a full detector simulation and by the 95\% CL estimation procedure (outlined in app.~\ref{Appendix_limitsetting}) used instead of a thorough HistFitter treatment. As our results are consistently more conservative than those of the ATLAS analysis, we consider our approach to be acceptable.

\begin{table}[!htbp]
\centering
\begin{tabular}{c|c|c|c}
 \hline
 \hline
 $\Mstar^{\tiny gen}$ & $\Mstar^{95\%\mathrm{CL}}$ [GeV] & $\Mstar^{95\%\mathrm{CL}}$ [GeV] & Difference \\
 $[$TeV$]$ & (ATLAS) & (this work) & $[\%]$ \\
 \hline
0.05 & 91 & 89 & 2.16 \\
0.3 & 1151 & 1041 & 7.3 \\
0.6 & 1868 & 1535 & 11.8 \\
1 & 2225 & 1732 & 12.0 \\
3 & 1349 & 1072 & 6.8 \\
6 & 945 & 769 & 8.5 \\
10 & 928 & 724 & 10.6 \\
30 & 914 & 722 & 9.6 \\
 \hline
 \hline
\end{tabular}
\caption{Comparison of the 95\% CL upper limits on $\Mstar$ from this work and from the ATLAS \monojet analysis \cite{Aad:2015zva}. The limits are for an $s$-channel vector mediator model with $\mX = $ 50 GeV and $\Gamma = \Mmed/8\pi$, and for the process $pp \rightarrow \chi \bar{\chi} + 1, 2j$ with QCUT = 80 GeV. Note that $\Mstar^{\tiny gen}$ is the input suppression scale.}
\label{M_star_limits_monojet}
\end{table}

\subsection{Mono-$Z$(lep) channel}
\label{monoZ_validation}

The ATLAS \monoZ results include an upper limit on the coupling $\gqX$ for a $t$-channel SiM analogous to our $tS$ model, and so it is this model which we use to validate our signal generation and selection procedures. Note that the following differences exist: the ATLAS model includes just two mediators ($up$- and $down$-type) where we consider six, the DM particle is taken to be Majorana where we assume Dirac, and the couplings $g_{t,b \chi}$ are set to zero where we have universal coupling to all three quark generations.

\begin{table}
\begin{center}
\begin{tabular}{ c | c | c | c | c }
\hline
\hline
$\mX$ & $\Mmed$ & $\gqX^{95\%\mathrm{CL}}$ & $\gqX^{95\%\mathrm{CL}}$ & Difference \T \\
$[$GeV$]$ & $[$GeV$]$ & (ATLAS) & (this work) & $[\%]$ \B \\
\hline
10 & 200 & 1.9 & 2.0 & 5.3 \T \\
 & 500 & 2.8 & 3.2 & 14.3 \\
 & 700 & 3.5 & 4.4 & 25.7 \\
 & 1000 & 4.5 & 5.2 & 15.6 \\
200 & 500 & 3.4 & 4.0 & 17.6 \T \\
 & 700 & 4.2 & 4.5 & 7.1 \\
 & 1000 & 5.2 & 5.3 & 1.9 \\
400 & 500 & 5.5 & 5.7 & 3.6 \T \\
 & 700 & 6.1 & 6.5 & 6.6 \\
 & 1000 & 7.2 & 7.4 & 2.8 \\
1000 & 1200 & 23.3 & 24.1 & 3.4 \T \B \\
\hline
\hline
\end{tabular}
\end{center}
\caption{Comparison of the 95\% CL upper limit on $\gqX$ from this work and from the ATLAS \monoZ analysis \cite{Aad:2014monoZlep}. The limits are for a variant of the $t$-channel scalar mediator model with Majorana DM for the process $pp \rightarrow \chi \bar{\chi} + Z (\rightarrow e^{+}e^{-}/\mu^{+}\mu^{-})$.}
\label{tab:monoZvalidation}
\end{table}

Table \ref{tab:monoZvalidation} shows the 95\% CL upper limits on $\gqX$ that we calculate using our own generation procedure (and the values in table~\ref{tab:sigmalim_monoZ}), compared with the limits taken from the ATLAS analysis. Also shown is the difference as a percentage of the ATLAS limit. We see reasonable agreement; most of the 11 points in parameter space are within 10\% of the ATLAS limits, and all are within 26\%. Additionally, our results are consistently more conservative, which again is to be expected given the less sophisticated nature of our generation procedure. As in the case of the \monojet validation, the differences stem from the use of $p_{\mathrm{T}}$ smearing applied to the leptons (rather than a full reconstruction simulation) and from the simplified treatment of systematics; we also obtained $\sigma \times \mathcal{A} \times \epsilon$ independently using the public results.

\subsection{Mono-$W/Z$(had) channel}
\label{monoWZ_validation}

The event generation and selection procedures for the \monoWZ channel are validated via reproduction of the ATLAS limits on $\Mstar$ for the D5 and D9 effective operators with $\mX = $ 1 GeV, using the upper limits on $\sigma \times \mathcal{A} \times \epsilon$ listed in table~\ref{tab:sigmalim_monoWZ}. We see agreement within 12.5\% and 7.4\% respectively, where the ATLAS limits are consistently stronger, as shown in table \ref{tab:monoWZvalidation}. The relative sizes of the discrepancies are expected given that only low-$\met$ limits are available for the D5 operator while we use the high-$\met$ signal region in our recast. Note that a general discrepancy of a few percent is expected for both operators for the reasons discussed in sections \ref{monojet_validation} and \ref{monoZ_validation}, and also because we use a cut-and-count approach while the ATLAS limits are extracted using a shape-fit. Furthermore, the ATLAS limits are quoted at 90\% CL while ours are calculated at 95\% CL.

\begin{table}[!h]
\begin{center}
\begin{tabular}{ c | c | c | c | c }
\hline
\hline
EFT operator & $\mX$ & $\Mstar^{90\%\mathrm{CL}}$ $[$GeV$]$ & $\Mstar^{95\%\mathrm{CL}}$ $[$GeV$]$  & Difference \T \\
&$[$GeV$]$ & (ATLAS) & (this work) & $[\%]$ \B \\
\hline
D9 & 1 & 2400 & 2221 & 7.4 \\
D5 & 1 & 570 & 499 & 12.5 \\
\hline
\hline
\end{tabular}
\end{center}
\caption{Comparison of the 95\% CL upper limits on $\Mstar$ from this work and from the ATLAS \monoWZ analysis \cite{Aad:2013monoWZ}. The limits correspond to the process $pp \rightarrow \chi \bar{\chi} + W/Z$ ($\rightarrow jj$).}
\label{tab:monoWZvalidation}
\end{table}


\begin{thebibliography}{99}
\input{References.tex}
\end{thebibliography}
\end{document}
