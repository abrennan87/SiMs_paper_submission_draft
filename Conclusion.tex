In this paper we have examined a set of three simplified dark matter models, extracting constraints from ATLAS Run I mono-$X$ plus missing energy searches featuring the associated production of a mono-jet, mono-$Z(\rightarrow$ leptons), or mono-$W/Z (\rightarrow$ hadrons). We explored a phase space where both the DM and mediator masses span $\mathcal{O}$(GeV) to $\mathcal{O}$(TeV), and considered ratios of $\gX / \gq$ of 0.2, 0.5, 1, 2 and 5 in the $s$-channel models.

Rather than setting limits in the $\Mmed - \mDM$ plane for a fixed value of the coupling strength, we instead constrained the coupling strength as a function of both $\Mmed$ and $\mDM$ in a 3D plane. Whilst this approach necessitates the introduction of some approximations, it also allows for a thorough examination of the interplay between the DM production cross-section and the free parameters of the models.

As expected, the \monojet channel is found to yield the strongest limits on vector and axial-vector SM and DM couplings to a vector mediator exchanged in the $s$-channel. This channel is also found to perform well for small values of $\gX$. The limits obtained in the \monoZ channel, in comparison, are generally weaker by a factor of a few, while the \monoWZ results are weaker again. This is partly due to our conservative estimations of the systematic uncertainties and partly due to limited statistics resulting from a harder $\met$ selection cut. The width effects associated with the $t$-channel exchange of an SU(2) doublet scalar mediator are observed to vanish in both the \monoZ and \monoWZ channels, greatly simplifying the analysis of this model and confirming these as straightforward and competitive channels for future collider DM detection.

Where the axial-vector model is not excluded by perturbative unitarity requirements, we find the coupling limits to be on par with those of the vector model within each analysis channel. Weaker limits are found for the $t$-channel model, a result of cross-section suppression not present in the $s$-channel models.

Finally, we compared our limits to constraints from relic density and direct detection; although each search is subject to a different set of assumptions, this demonstrates the complementarity and impressive reach of simplified models as a tool for the interpretation of collider DM searches. We eagerly await the improved constraints expected from Run II of the LHC.
