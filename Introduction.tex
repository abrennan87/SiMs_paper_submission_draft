Simplified models have emerged as a powerful tool for the interpretation of collider, direct and indirect detection signals of dark matter (DM). Previously, ATLAS and CMS searches for DM were conducted within the context of both Effective Field Theories (EFTs) \cite{Aad:1363019, ATLAS-CONF-2012-147, CMS-PAS-EXO-12-048, Abdallah:1472683} and full UV-complete theories such as Supersymmetry \cite{Aad:2012ms, Aad:2012fqa, Aad:2014wea, SUSY_official_paper}. The latter approach, though well-motivated, is typified by a broad parameter space and generally yields results which are insensitive to the wider class of DM models. EFT constraints, in comparison, are applicable to a broad range of models and rely on the specification of only a small set of parameters, namely the suppression scale, $\Mstar$, and the DM mass, $\mDM$ \cite{DMCons2}.

In the EFT framework, interactions between the dark and Standard Model (SM) sector are parametrised by a set of higher-dimensional effective operators that arise when the mass of the mediating particle is assumed to be significantly larger than the momentum transferred in a given interaction. Where this condition is not fulfilled, the EFT prescription can produce constraints which detour dramatically from those of the associated UV-complete model \cite{Bai:2010hh, DMCons2, Fox:2011fx, Graesser:2011vj, An:2011ck}. This is less important in direct detection experiments where the momentum transferred in the scattering of DM particles with heavy nuclei is generally of the order of tens of MeV \cite{EFTDM, DMCons3}, or in indirect searches where the annihilations of non-relativistic DM particles in the galactic halo occur with momentum transfers of order $\mDM$. However, for hadron collider searches, where the accessible center of mass energy of two colliding baryons may be sufficient to produce the mediator on-shell, the range of validity of the EFT prescription is significantly diminished. Indeed, recent works have quantitatively shown that the EFT approach can break down as a valid interpretation of data collected during the $\sqrt{\hat{s}} =$ 8 TeV Run I of the Large Hadron Collider (LHC) \cite{Buchmueller:2013dya,ValidEFT, ValidEFT_part2, ValidEFT_part3}. In light of this, simplified models have been investigated as an alternative approach.

In a nutshell, a simplified model (SiM) arises when the heavy mediator which was integrated out in the EFT framework is reintroduced. This must be done with caution in order to ensure that the phenomenology of the resultant SiM is representative of a realistic UV-complete theory of DM.
%
Like EFTs, SiMs facilitate the comparison of results obtained in  different avenues of dark matter study \cite{DiFranzo:2013vra, Buckley:2014fba} and are defined by a relatively small set of parameters - often $\mDM$, the mass of the mediator $\Mmed$, and the SM-mediator and DM-mediator coupling strengths, $\gq$ and $\gX$ (or $\gqX$ in the case of a single, SM-DM-mediator coupling). This increased parameter space makes it challenging to scan as wide a range of models as can be done with EFTs.
Relative to EFTs, constraints calculated within the context of a SiM are valid across a broader energy range. 
%
Whilst EFTs remain a useful tool if used with caution, simplified models have become the preferred tool for the interpretation of collider DM searches \cite{DM_MET_LHC, DMOxfordReport, DMForumReport, Harris:2014hga,Buchmueller:2014yoa}.


In this paper, we examine a phenomenologically distinct set of SiMs. In particular, we place constraints on the SiMs corresponding to the simplest UV-completions of the D5 (vector) and D8 (axial-vector) effective operators in the $s$-channel\footnote{The D5 and D8 operators form a nice starting point in the analysis of SiMs as they have been studied exhaustively in the past (see refs.~\cite{Aad:1363019, ATLAS-CONF-2012-147, CMS-PAS-EXO-12-048, Buckley:2013jwa, Abdallah:1472683, MonoX, ValidEFT, ValidEFT_part2, ValidEFT_part3} among others). This attention is motivated by the fact that collider limits for the D5 (D8) operator can be readily transformed into limits on spin-independent (spin-dependent) DM-nucleon scattering and vice versa.}. We also include a case in which a scalar mediator is exchanged in the $t$-channel, motivated by its analog of squark exchange in Supersymmetry. In the heavy mediator limit, the operator describing this model can be rearranged via a Fierz transformation into a combination of operators D5 to D8.

The models are constrained using public results from \monoX + missing transverse energy ($\met$) searches conducted by the ATLAS Collaboration. Specifically, we focus on searches where $X$ is either a parton (manifesting in the detector as a narrow-radius jet), a leptonically-decaying $Z$ boson (manifesting as two opposite-sign same-flavor leptons), or a hadronically-decaying $W$ or $Z$ boson (manifesting as a large-radius jet). The purpose of this work is to strengthen existing SiM limits using the full 20.3 $fb^{-1}$ of Run I ATLAS data, and to explore an enhanced phase space with respect to the mediator and DM masses and the relative strength of the couplings to the visible and dark sectors. We choose to treat the mediator width as the minimal value naturally arising, which is more realistic than a fixed width. Lastly, we provide a cross-check and comparison of the performance of the three targeted collider detection channels, and compare against relic density and direct detection constraints.

The remainder of the paper is organised as follows. Section \ref{sec:sec2} contains a compendium of the SiMs chosen for analysis and the associated collider phenomenology. Section \ref{sec:sec3} outlines the techniques used to recast \monoX + $\met$ limits on the visible cross-section for any new physics process into constraints on SiMs, and specifically on the couplings $\gq$ and $\gX$. Lastly, our results are presented in section \ref{sec:sec4} along with a discussion of the implications of this work. Appendices \ref{Appendix_limitsetting} and \ref{Appendix_validation} include  details of the limit setting and analysis validation procedures.
